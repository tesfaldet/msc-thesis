\begin{figure}[t]
\begin{center}
\begin{tabular}{ >{\centering\arraybackslash} m{0.16\textwidth} || >{\centering\arraybackslash} m{0.80\textwidth} }
{target (\path{leaves})} & 
\showtexture{leaves/frame_} \\
\hline \hline
{Funke \etal \cite{funke2017}} &
\showtexture{leaves_funke/frame_} \\
\hline
{two-stream model} & 
\showtexture{leaves_output/frame_} \\
\end{tabular}
\end{center}
\vspace{-0.45cm}
\caption[Qualitative comparison with Funke \etal's \cite{funke2017} model]{Qualitative comparison with Funke \etal's \cite{funke2017} model on one of the sequences used in their work. Note that this sequence does not follow the thesis' assumption of a dynamic texture, in the sense that the appearance and/or dynamics are not spatiotemporally homogeneous.
(top row) Target sequence.
(middle row)
Dynamic texture synthesis when using Funke \etal's model. The model fails to capture the up-right motion of the leaves.
(bottom row)
Dynamic texture synthesis when using the proposed two-stream model. The up-right motion of the leaves is captured. Results are best viewed in video.
}
\label{fig:funke_comparison}
\end{figure}

