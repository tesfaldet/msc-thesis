\section{Empirical evaluation}

The goal of (dynamic) texture synthesis is to generate 
samples that are indistinguishable from the real input target
texture by a human observer. To qualitatively evaluate the proposed model, I will apply my dynamic texture synthesis process 
to a wide range of textures which will be selected from the 
DynTex \cite{peteri2010} database and others I will collect in
the wild. I will group and analyze the results in terms of appearance and dynamics characteristics. To analyze the effect of appearance on the quality of results, textures will be grouped as either regular/near-regular (\eg, periodic tiling and brick wall), irregular (\eg, a field of flowers), or stochastic/near-stochastic (\eg, tv static or water). This follows the taxonomy previously proposed by Lin et al.\ \cite{lin2006quantitative}. Similarly for dynamics, textures will be grouped as either spatially-consistent (\eg, closeup of rippling sea water) or spatially-inconsistent (\eg, rippling sea water juxtaposed 
with translating clouds in the sky). In addition, I will perform a qualitative comparison with the dynamic texture synthesis approach of Funke et al.\ \cite{funke2017} and Xie et al.\ 
\cite{xie2017synthesizing}. To quantitatively evaluate the proposed model, I will 
quantify perceptual differences between dynamic textures synthesized using my method and real dynamic textures.

To validate that the appearance stream will not inadvertently include dynamics information, I will generate videos using the appearance loss only, \ie,
Eq.\ \ref{eq:apploss}. The produced frames should be valid textures but
with no coherent dynamics present. Similarly, to validate that the dynamics stream will not 
inadvertently include appearance information, I will generate videos
using the dynamics loss only, \ie, Eq.\ \ref{eq:dynloss}. The resulting frames should have no visible appearance.