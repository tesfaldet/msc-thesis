\chapter{Related work \todomatthew{unfinished}}

There are two general approaches that have dominated the texture
synthesis literature: non-parametric sampling approaches that
synthesize a texture by sampling pixels of a given source texture
\cite{efros1999,kwatra2003graphcut,schodl2000,wei2000}, and 
statistical parametric models.
As the proposed approach is an instance of a parametric model, this thesis 
will focus on these parametric approaches.

The statistical characterization of visual textures was introduced
in the seminal work of Julesz \cite{julesz1962}.
He conjectured that particular statistics of pixel intensities
were sufficient to partition spatial textures into metameric (\ie,
perceptually indistinguishable) classes.
Later work leveraged this notion for texture synthesis
\cite{heeger1995pyramid,portilla2000parametric}.
In particular, inspired by models of the early stages of visual 
processing, statistics of (handcrafted) multi-scale oriented filter 
responses were used to optimize an initial noise pattern 
to match the filter response statistics of an input texture.
More recently, Gatys \etal \cite{gatys2015} demonstrated
impressive results by replacing the linear filter bank with a
ConvNet that, in effect, served as a proxy for the ventral visual
processing stream.
Textures are modelled in terms of the correlations between filter 
responses within several layers of the network.
In subsequent work, this texture model was used in image style
transfer \cite{gatys2016image}, where the style of one image was
combined with the image content of another to produce a new image.
Ruder \etal \cite{ruder2016} extended this model to video by using
optical flow to enforce temporal consistency of the
resulting imagery.

Variants of linear autoregressive models have been studied
\cite{szummer1996,doretto2003} that jointly model appearance and
dynamics of the spatiotemporal pattern.
More recent work has considered ConvNets as a basis for modelling 
dynamic textures.
Xie \etal \cite{xie2017synthesizing} proposed a spatiotemporal
generative model where each dynamic texture is modelled as a random
field defined by multiscale, spatiotemporal ConvNet filter responses
and dynamic textures are realized by sampling the model.
Unlike my current work, which assumes pretrained fixed networks,
this approach requires the ConvNet weights to be trained using the
input texture prior to synthesis.

A recent preprint \cite{funke2017} described preliminary 
results extending the framework of Gatys \etal \cite{gatys2015} 
to model and  synthesize dynamic textures by computing a Gram 
matrix of filter activations over a small temporal window.
In contrast, the proposed two-stream filtering architecture is more 
expressive as the dynamics stream is specifically tuned to 
spatiotemporal dynamics.
Moreover, the factorization
in terms of appearance and dynamics enables a novel form of
style transfer, where the dynamics of one pattern are 
transferred to the appearance of another to generate an
entirely new dynamic texture.
This work is the first to demonstrate this form of style transfer.

The recovery of optical flow from temporal imagery has long been
studied in computer vision.
Traditionally, it has been addressed by handcrafted approaches
\eg, \cite{horn1981,lucas1981,revaud2015epicflow}.
Recently, ConvNet approaches \cite{dosovitskiy2015,ranjan2017,ilg2017,yu2016}
have been demonstrated as viable alternatives.
Most closely related to my approach are energy models of visual
motion \cite{adelson1985spatiotemporal,heeger1988,simoncelli1998,nishimoto2011,derpanis2012spacetime,konda2014}
that have been motivated and studied in a variety of contexts,
including computer vision, visual neuroscience, and visual
psychology.
Given an input image sequence, these models consist of an
alternating sequence of linear and non-linear operations that yield
a distributed representation (\ie,  implicitly coded) of pixelwise
optical flow.
Here, an energy model motivates the
representation of observed dynamics which will then be encoded
as a ConvNet. Significantly, a completely analytically-defined
oriented energy ConvNet model provides the current state-of-the-art
for the related task of dynamic texture recognition \cite{hadji2017}.