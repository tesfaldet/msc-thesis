\section{Experimental procedure}\label{sec:experimental_procedure}
Provided here are the experimental details of the user study
using Amazon Mechanical Turk (AMT). Experimental trials were
grouped into batches of Human Intelligence Tasks (HITs) for
users to complete. Each HIT consisted of 59 pairwise
comparisons between a synthesized dynamic texture and its target.
Users were asked to choose which texture appeared more realistic
after viewing each texture independently for an exposure time (in seconds) 
sampled randomly from the set \{0.3, 0.4, 0.6, 1.2, 2.4, 3.6, 4.8\}.
Note that 12 frames of the dynamic texture corresponds to 1.2 seconds, \ie, 10 frames per second. Since the dynamic textures were collected from various sources with varying framerates, a canonical framerate was chosen for all input and synthesized dynamic textures. 10 frames per second was chosen to allow sufficient viewing time (1.2 seconds) while maintaining easily distinguishable dynamics.
Before viewing a dynamic texture, a centred dot is flashed twice to
indicate to the user where to look (left or right).
To prepare users for the task, the first three comparisons 
were used for warm-up, exposing them to the shortest (0.3s), 
median (1.2s), and longest (4.8s) durations.
To prevent spamming and bias, the experiment was constrained as 
follows:
\begin{enumerate}
	\item Users could make a choice only after both dynamic textures were shown;
	\item The next texture comparison could only be made after a decision was made for the current comparison;
	\item A choice could not be changed after the next pair of dynamic textures were shown;
	\item Users were each restricted to a single HIT.
\end{enumerate}
Obvious unrealistic dynamic textures were synthesized by 
terminating synthesis early (100 iterations) and were used as sentinel tests. An example is shown in Fig.\ \ref{fig:sentinel}.

\begin{figure}[t]
\begin{center}
\begin{tabular}{ >{\centering\arraybackslash} m{0.16\textwidth} || >{\centering\arraybackslash} m{0.80\textwidth} }
{target (\path{water_1})} & 
\showtexture{water_1/frame_} \\
\hline \hline
{synthesized} &
\showtexture{water_1_output/frame_} \\
\hline
{sentinel} & 
\showtexture{water_1_sentinel/frame_} \\
\end{tabular}
\end{center}
\vspace{-0.45cm}
\caption[Example of a sentinel dynamic texture for the user study]{Example of a sentinel dynamic texture for the user study.
(top row) Target dynamic texture.
(middle row)
An obviously-unrealistic synthesized dynamic texture (a sentinel example). Achieved by terminating synthesis early (100 optimization iterations).
(bottom row)
Dynamic texture synthesized after $6,000$ optimization iterations.
}
\label{fig:sentinel}
\end{figure}

Three of the 59 pairwise comparisons were sentinels and results from
users who gave incorrect answers on any of the sentinel
comparisons were not used. The left-right order of textures within a pair,
display order within a pair, and order of pairs within a HIT, were randomized.
An example of a HIT is shown in a video included with the supplemental material: \path{HIT_example.mp4}.

Users were paid \$2 USD per HIT, and were required to have at least
a 98\% HIT approval rating, greater than or equal to 5000 HITs
approved, and to be residing in the US. Results were collected from 200 unique 
users to evaluate the final model (which uses the ``Concat layer'') and another 200 to evaluate the baseline model (which uses the ``Flow decode layer'').